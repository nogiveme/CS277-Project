\section{Background}
\label{sec:background}
\subsection{Financial Risk}
The accurate assessment and proactive mitigation of financial risk are paramount for ensuring market stability and sound investment strategies. Financial risk prediction is fundamentally defined as the quantitative process of estimating the likelihood of an adverse financial outcome (e.g., asset impairment, corporate default) within a given period. Academic frameworks classify this risk into several distinct categories, essential for comprehensive risk management~\cite{risk_classification}.
Market Risk. This quantifies potential losses due to adverse price fluctuations (e.g., equity, interest rate, or foreign exchange rates). It is characterized by high-frequency dynamics and sensitivity to real-time market events and investor sentiment.
Liquidity Risk. This is the risk that a firm cannot meet its short-term debt obligations without incurring substantial losses, typically due to an inability to easily sell assets or raise cash. It has a mid-to-high frequency component, driven by trading volume and market depth.
Credit Risk. This addresses a counterparty's fundamental inability or unwillingness to meet its contractual debt obligations. It is characterized by its low-frequency nature, often correlating with statutory financial reporting cycles, and reflects the counterparty's intrinsic solvency.
Business Risk. This refers to the fundamental risk associated with a company's operations, reflecting the uncertainty of future profits. It is typically assessed via low-frequency fundamental metrics like cost structure and revenue volatility.
Investment Risk. This broadly encompasses the chance that actual investment returns will be different from expected returns. It often integrates aspects of the risks mentioned above when evaluating portfolio performance.
This project is specifically designed to enhance the prediction of firm-level risk by capturing signals related to Credit Risk and Market Risk from diverse, multi-modal data sources.

\subsection{Static and Time Series Data}
Effective risk modeling necessitates the integration of heterogeneous data, a core challenge this project addresses. We categorize the inputs used in this research based on their temporal nature: Static Data and Time Series Data.
Static Data. This modality primarily comprises Fundamental Data derived from corporate financial statements. These inputs, such as debt-to-equity ratios and profitability metrics, are updated at a low frequency (quarterly or annually). They provide a deep, structural snapshot of a firm’s intrinsic value and long-term operational health.
Time Series Data. This modality includes Historical Trading Data sourced from stock exchanges (prices, volume, volatility). Characterized by its high frequency (tick, minute, or daily), this data captures continuous market dynamics, real-time sentiment, and instantaneous liquidity.
The fundamental challenge in utilizing these streams jointly is the issue of fusion: successfully bridging the gap between the static, structural nature of fundamental data and the dynamic, high-frequency nature of time series data to achieve a unified, comprehensive risk profile.

\subsection{Transformer}
The development of the Transformer architecture marks a paradigm shift in sequential data processing, offering significant advantages over traditional recurrent neural networks (RNNs) for complex, long-range financial dependencies. Central to its superior performance is the Self-Attention Mechanism, which allows the model to globally weigh the relevance of all input elements (e.g., historical data points) to a single element being processed, regardless of their distance. This non-sequential processing capability is crucial for financial data, which is often characterized by non-linear, multi-scale, and long-range temporal correlations. Consequently, the Transformer has emerged as a state-of-the-art solution across various FinTech applications, including: high-frequency algorithmic trading, time series forecasting, and automated credit scoring. In the context of risk prediction, the Transformer’s ability to efficiently capture complex relationships between distant historical market events and current fundamental data offers a robust foundation for building high-accuracy, context-aware risk models.

