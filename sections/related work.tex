\section{Related Work}
\label{sec:related work}
The application of machine learning (ML) in financial risk prediction has led to two primary research trajectories: Fundamental-based Machine Learning Research, which focuses on analyzing structured accounting data for solvency assessment, and Market-based Time Series Modeling Research, which models high-frequency data for short-term dynamics.

\subsection{Fundamental-based Method}
This research trajectory centers on Fundamental-based Machine Learning Research, aiming to leverage detailed financial statement data (e.g., balance sheets and income statements) to assess corporate failure risk. The analysis focuses on metrics reflecting operations, revenue, profit, and profit margins~\cite{gu2020empirical}~\cite{ahbali2022identifying}~\cite{wang2023sparsity}. These studies demonstrate superior classification performance compared to traditional statistical models. However, the predictive horizons of these models are inevitably constrained by the low frequency of the input data (quarterly reports). Consequently, models based purely on fundamental data are slow to react to rapidly emerging risk factors or sudden shifts in market confidence, limiting their utility for instantaneous risk management.

\subsection{Market-based Method}
In contrast to the low-frequency constraints of fundamental analysis, a parallel body of literature has focused on Market-based Time Series Modeling Research to capture the temporal dynamics of high-frequency trading data~\cite{random_forest}~\cite{svm}. This approach utilizes advanced sequence models, including architectures such as Recurrent Neural Networks (RNN), Long Short-Term Memory (LSTM), and Gated Recurrent Units (GRU), which have proven effective in forecasting short-term volatility, market momentum, and aggregate sentiment. While these models offer high real-time responsiveness, they typically lack the critical constraint of a company's fundamental value. Predictions derived solely from market signals are prone to amplifying transient noise, technical biases, or herd behavior, often failing to distinguish between fleeting market overreaction and genuine, fundamental solvency deterioration.

\subsection{Existing Multi-modal Fusion Attempts and The Critical Gap}
While both research paths offer valuable, yet incomplete, insights, the logical evolution involves combining these heterogeneous data sources. Existing Multi-modal Fusion Attempts have been explored, but these studies reveal a Critical Gap in the literature. Early fusion attempts are often limited to simple feature concatenation, where extracted features from both modalities are merely joined before being fed into a final classification layer. This approach is shallow; it fails to learn the complex, non-linear interaction and synergistic effects between the two data streams . Specifically, these models struggle with: (1) Developing a sophisticated deep fusion mechanism to align and weigh multi-scale features appropriately, and (2) Achieving adequate model interpretability , which is crucial in regulated financial environments, making it difficult to ascertain whether a risk signal originates from underlying financial distress or panic-driven market dynamics.

The absence of an innovative, deeply integrated, and interpretable multi-modal deep learning architecture capable of harmonizing high-frequency market dynamics with low-frequency fundamental value. This project aims to fill this void by developing such a framework to significantly enhance the robustness and diagnostic capability of financial risk prediction.