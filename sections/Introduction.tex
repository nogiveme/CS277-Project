\section{Introduction}
\label{sec:Introduction}
The stability of modern financial systems depends heavily on the accurate prediction and effective management of financial risks.
These risks—primarily credit risk, market risk, and liquidity risk—are interdependent and evolve dynamically under rapidly changing market environments.
With the growing complexity of global markets and the massive availability of real-time financial data, there is an increasing need for intelligent and data-driven methods that can monitor and anticipate risk fluctuations in a timely and interpretable manner.

Traditional econometric models, such as Vector Autoregression (VAR) or GARCH-type frameworks, often struggle to capture the nonlinear dependencies and high-frequency dynamics inherent in financial markets.
Recent advances in machine learning have introduced deep neural models capable of extracting complex patterns from large-scale financial data.
However, most existing models still treat different data modalities—such as time-series trading data (e.g., daily returns, price volatility, transaction volume) and static fundamental indicators (e.g., leverage ratio, profitability, cash flow, macroeconomic metrics)—as isolated information sources.
This limitation hinders the model’s ability to reflect the joint effects of short-term market turbulence and long-term structural fundamentals.

In this context, Transformer-based architectures have emerged as a powerful paradigm for sequential data modeling, offering superior capability in capturing long-range temporal dependencies through self-attention mechanisms.
Their non-recurrent structure allows efficient parallel computation and interpretable feature weighting, making them particularly suitable for high-frequency and heterogeneous financial data.
Nevertheless, integrating static and dynamic information streams within a unified Transformer framework remains an open challenge due to differences in sampling frequency, feature representation, and statistical characteristics.

To address these challenges, this project proposes a multimodal Transformer-based framework that integrates dynamic market signals with static financial fundamentals for comprehensive financial risk prediction and cross-risk correlation analysis.
The model aims to jointly evaluate credit, market, and liquidity risks by fusing information from diverse temporal and structural data sources.
By leveraging multi-head attention mechanisms, the system will learn hierarchical feature dependencies across modalities and risk dimensions, enabling interpretable identification of systemic interactions.
This integrated approach is expected to enhance the robustness and diagnostic capability of financial risk assessment, providing valuable insights for market stability, regulatory supervision, and investment decision-making.